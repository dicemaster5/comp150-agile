% Please do not change the document class
\documentclass{scrartcl}

% Please do not change these packages
\usepackage[hidelinks]{hyperref}
\usepackage[none]{hyphenat}
\usepackage{setspace}
\doublespace

% You may add additional packages here
\usepackage{amsmath}

% Please include a clear, concise, and descriptive title
\title{How long should a sprint last\\when developing games?}

% Please do not change the subtitle
\subtitle{COMP150 - Agile Development Practice}

% Please put your student number in the author field
\author{1703086}

\begin{document}

\maketitle

\abstract{\textit{Abstract: Team working frameworks such as Scrum are popular within the games industry at the moment, they come with great solutions to working coherently in teams. This essay will be looking at Scrum and the way teams should deal with sprint lengths to see if there is an ideal time length for teams to work with. It will also look into the idea of weekly sprints, which are rarely proposed by most sources talking about scrum, and find out why and if they could be more ideal for developing games compared to other sprint lengths.}}

\section{Introduction}
\iffalse
Write your introduction here. A brief introduction is recommended, which should outline key details of the chosen topic and the reviewed papers, motivate the work, and provide a roadmap of key points to the reader. The motivation is quite important here, as essays should have a contribution (i.e., what is the point of the essay, and what does the reader take away from the essay) and the link between the motivation (in the introduction) and the contribution (in the conclusion) should be made clear.
\fi
This essay will be about Scrum methodology and the importance of the work cycles that are practised within Scrum, known as Sprints, and more importantly how long a sprint should last.\\
It will also be reviewing multiple different sources that discuss varying opinions on the length of sprints and what the ideal length should be depending on the scale of the project.\\
It will start of by explaining the importance of Sprints in Scrum for games development and then go into why Sprints can be of different lengths to then argue on whether there is an ideal length for sprints to last, it will further go on to arguing that 1 week long sprints could be quite a ideal time lengths for small teams, and it will finally end with a conclusion of the argument and essay.


\section{So let's talk about sprints}
\iffalse
Write the main body of your essay here. Add more sections if appropriate. You may choose to write about each of your three papers in its own section, or you may choose a different structure. Either way, remember that you are being assessed on technical insight and analysis: it is not enough to merely summarise the contents of the three papers. You must demonstrate the ability to make inferences beyond what is written in the papers, and to draw the three papers together into a single coherent narrative.

Your essay must make a clear recommendation, in terms of which of the three techniques you have reviewed is the best according to whichever metric or metrics you feel is most appropriate. You must justify your choice, backing it up with empirical evidence. However remember that an academic essay is not a murder mystery: you should already have briefly discussed your recommendation in the introduction and in other parts of the essay. Do not save it for a grand reveal at the end.
\fi
Developing digital games is hard and it gets even harder when you need to develop games with a team, so much more can go wrong when multiple people are working together on the same project, this is why a lot of people within the games industry practice Agile development methods; such as Scrum.
\\~\\
Scrum has an incremental development process known as Sprints, Sprints are pre-planned time periods in which the team has decided on what they will work on during the sprint. Sprints are a very important part of the Scrum methodology as they give a specific time limit for the team to work on tasks to implement into the project \cite{five} \cite{eight}.\\
At the end of a Sprint, a sprint review meeting occurs in which the team gets together to discuss what happened during the sprint and update their project to the latest build of the game with all the new features and fixes that were made during the sprint\cite{eight}.\\
Some developers like to use sprint reviews as a way of updating their fans on what they have been working on lately for the game, other developer who have early access games might even update their game after each sprint to give the fans regular new features and bug fixes in order to keep the game fresh for the players.
\\~\\
This is what the Agile manifesto says about sprints ``Deliver working software frequently, from a couple of weeks to a couple of months, with a preference to the shorter timescale.``\cite{three}.
\\~\\
So Sprints are an important part of the development cycle for games and they need to be well planned and organized in order for the team to work efficiently on a game.

\section{So how long should a sprint last?}
A lot of sources say that sprints should last 2-4 weeks depending on the team and the scale of the project \cite{two} \cite{four} \cite{five}. Some sources that have been explored also suggest that shorter sprints are generally better for software and games development \cite{three}. Shorter sprints generally work in favour for the Agile Scrum methodology for being short iterative steps towards development. A lot of teams have mentioned that they generally have 2 week long sprints as their main sprint duration as it gives a good time frame for developers to work on the scrum backlog. Any more than 4 weeks is usually considered too long and unfitting for the agile philosophy as it means the developers aren't iterating as much.
\\~\\
Some teams like to change up the sprint duration during development because they might be at a different stage that requires more or less time for sprint runs. So for example a team could start off developing a game with 3 week long sprints and then change to 2 week long sprints once they have finished the alpha stage of the game and can now work on implementing more content instead of working on the overall framework.

\section{The idea of 1 week sprints?}
1 week long sprints are rarely mentioned and are a fairly controversial idea, and for good reasons. This Essay wants to shed some light onto the positive and negatives of the weekly sprint and argue for why it's actually a good idea and that game dev teams should practice it more.
\\~\\
So let's start off with the negatives:\\
Some sources say that 1 week sprints are too short and don't give enough time for teams to work everything they wanted to work on from their sprint planning, other sources say that 1 week sprints aren't good because it means a loss of time from having to do more sprint planning meetings and review meetings.
\\~\\
Now let's look at the positives:\\
1 week or 5 working days is actually quite a long time when you think about it and a lot can be achieved during that time frame.
An article by Dan Ackerson suggests that developers get sloppy and don't work as productive as productive as they could be in sprints that last more than a week and encourages the idea of 1 weeks sprints as it gets the team more focused on the sprint goals\cite{six}.
Rob Galanakis in his article ``Two weeks is the worst sprint length`` says that he prefers 1 week long sprints as they allow for more feedback and chances to improve, but he admits that you need a confident team in order to have successful weekly sprints\cite{seven}.

The weekly sprint idea doesn't get mentioned enough and really should, as it's a great way for experimenting different approaches to development cycles. But it's more of a preference at the end of the day, some teams might enjoy it more than others.


\section{Conclusion}
\iffalse
Write your conclusion here. The conclusion should do more than summarise the essay, making clear the contribution of the work and highlighting key points, limitations, and outstanding questions. It should not introduce any new content or information.
\fi
In the end the is no real ideal sprint length, the length of a sprint entirely depends on what the team is capable of doing depending on the scope and scale of a game development project. The team decides, they get to choose what will best suite their needs for developing a game.
\bibliography{references}
\bibliographystyle{ieeetran}


\end{document}
